\documentclass[]{article}
\begin{document}
	
\subsection{Vertically center beam to HMS}

Use the previous run, or if you are in the iteration process run at less than 10 μA current, use the Carbon optics target. Use a fast raster size of 1 by 1 mm2. The exact raster shape does not matter: the raster can even be off in this configuration. Take a short run (100K). Analyze and make an ntuple. Make the following plots: x vs y at the focal plane, xp vs yp at the target (you may have to use some Particle Id. and reconstruction cuts here!), and y at the target. You want to check the following: is the central leg of the ``spider" in x vs y at the focal plane straight? is the reconstructed y position close to y = 0? Is the central sieve slit hole close to (yp,xp) = (0,0)? If the xp position of the central sieve hole is close to 0, you probably are a bit off in vertical beam position after all, fix this. If you are far off (larger than 2 mr), check all your results carefully, did something go wrong in the vertical beam assignment? If the central leg of the ``spider" is close to vertical, you are close to having mid-plane symmetry for the spectrometer. You can vary the horizontal beam position a bit to check this. Note that the present quad alignment is such that the quad system is about 1 mm to the right of the line through the nominal pivot and the spectrometer angle, so the y position at the target can be a little bit negative, and the central leg of the spider can be slightly tilted. If the yp position of the central sieve hole is within 1 or 2 mr of the nominal zero position you are probably fine. The big uncertainty will be whether the targets are actually located at the nominal pivot (z = 0) position. If the target survey says otherwise, you expect (i) the central leg of the ``spider" not to be straight, (ii) the y reconstruction not to be perfect, and (iii) an offset in yp for the central sieve hole. If the three pieces of information are pretty much consistent with the survey assume you are done (note: the HMS sieve is at a distance of 1.66 meter of the target). You can consider checking this by using a hole target, or by rotating the spectrometer to a larger angle, and verifying that indeed things are consistent.
\end{document}
